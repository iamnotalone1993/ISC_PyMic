\label{conclusion}
In this article, we propose and develop pyMIC2 which is a NumPy-like library for deep learning frameworks run on Intel Xeon Phi Knights Corner coprocessor. This library is an extension of pyMIC that is a Python offloading module for Intel Xeon Phi Knights Corner coprocessor. The experimental results show that pyMIC2 not only outperforms compared with NumPy when considering them on two distinct hardware platforms with the theoretical performance, is but also able to be highly integrated into one popular deep learning framework Chainer with convincing performance. 

However, pyMIC2 still contains several limitations. First, we initially implement some fundamental functions which are sufficient to integrate it into Chainer in order to run ANNs with MNIST dataset. There are still enormous functions that need to be implemented to be capable of running other networks. As a result, we intend to implement other NumPy-like functions so as to apply to other deep learning networks. Second, pyMIC2 can currently be run on at most one Intel Xeon Phi Knights Corner coprocessor and hence, it does still not exploit fully systems containing more than one Intel Xeon Phi Knights Corner coprocessor. For that reason, we intend to enhance pyMIC2 so that it is able to be run such systems. Eventually, computation loads are performed completely on only Intel Xeon Phi coprocessor when using pyMIC2. Central Processing Units (CPUs) have still not been utilized to share the burden of on computation. Therefore, we intend to improve pyMIC2 so that it can be run on both CPUs and the coprocessor simultaneously to boost program performance. 